\documentclass[10pt]{scrartcl}
%
%%%%%%%%%%%%%%%%%%%%%%%%%%%%%%%%%%%%
%% Common preamble
%%%%%%%%%%%%%%%%%%%%%%%%%%%%%%%%%%%%
% PAGE
\usepackage{fullpage} % uncomment this when changing to jour/conf style
% FONTS
\usepackage{lmodern} % enhanced version of computer modern
\usepackage[T1]{fontenc} % for hyphenated characters and textsc in section title
\usepackage{microtype} % some compression
% MATH
\usepackage{amssymb}
\usepackage{listings}
\usepackage{mathtools} % contains amsmath which comes with align
\usepackage{amsthm} % newtheorem stuff
\usepackage{bm} % for bold math (use $\boldsymbol{}$)
% COLOR
\usepackage[usenames,dvipsnames]{color}
%%
% REFERENCING
\usepackage[square,sort&compress,numbers]{natbib} %sorts bibs when they are collectively cited
%%
% TABLES
\usepackage{multirow}
%\usepackage{ctable} % provides toprule, bottomrule, midrule
\usepackage{array} % new implementation of tabular & array with lots of enhancements
%%
% FIGURES
\usepackage{graphicx}
\usepackage{grffile} % to set right names of files
%%
% CAPTIONS
\usepackage{caption}
\usepackage{subcaption} % supersedes subfigure & subfloat. try using options
%%
% LISTS
\usepackage{enumitem}
% ALGO
\usepackage[ruled,linesnumbered]{algorithm2e}
%% comments & todo
\newcommand{\comment}[1]{{\color{red}#1}}
\usepackage[colorinlistoftodos]{todonotes}
\newcommand{\TODO}[1]{\todo[inline,color=red!10,size=\small]{#1}}
%% COMMANDS
\newcommand{\comp}[1]{\overline{#1}}  %{\widetilde{#1}}
\DeclareMathOperator{\Var}{Var}
\DeclareMathOperator{\Cov}{Cov}
\DeclareMathOperator{\bigo}{O}
\DeclareMathOperator{\bigom}{\Omega}
\DeclareMathOperator{\davg}{d_{avg}}
\DeclareMathOperator*{\argmax}{arg\,max}
\DeclareMathOperator*{\argmim}{arg\,min}
% Note: \deg is already defined
\newcommand{\expect}{\mathbb{E}}
%
\newcommand{\ceil}[1]{\left\lceil #1 \right\rceil}
\newcommand{\floor}[1]{\left\lfloor #1 \right\rfloor}
%
\newcommand{\reals}{\mathbb{R}}
\newcommand{\field}{\mathbb{F}}
\newcommand{\integers}{\mathbb{Z}}
%
\newtheorem{theorem}{Theorem}[]{\bfseries}{\itshape} 
\newtheorem{lemma}[theorem]{Lemma}{\bfseries}{\itshape}
\newtheorem{claim}[theorem]{Claim}{\bfseries}{\itshape}
\theoremstyle{definition}
\newtheorem{definition}[theorem]{Definition} % {\bfseries}{\itshape}
\newtheorem{observation}[theorem]{Observation} % {\bfseries}
\newtheorem{condition}[theorem]{Condition} % {\bfseries}{\itshape}
\newtheorem{note}[theorem]{Note} % {\bfseries}{\itshape}
% ROMAN NUMERALS
\makeatletter
\newcommand{\rmnum}[1]{\romannumeral #1}
\newcommand{\Rmnum}[1]{\expandafter\@slowromancap\romannumeral #1@}
\makeatother
% TWO VERSIONS
%% \usepackage{etoolbox}
%% \newtoggle{withappendix}
%% \toggletrue{withappendix} % comment this if you want journal version
%% Usage: iftoggle{withappendix}{}{}
%%
%% math operator
%% \DeclareMathOperator{\sgn}{sgn}
% CODEBOX
%% \usepackage[framemethod=tikz]{mdframed}
%% \newmdenv[linecolor=black!10,innerlinewidth=0pt, roundcorner=4pt,innerleftmargin=6pt,
%% font=\ttfamily,innerrightmargin=6pt,innertopmargin=6pt,
%% innerbottommargin=6pt,backgroundcolor=black!10]{codeblock}
%%%%%%%%%%%%%%%%%%%%%%%%%%%%%%%%%%%%
%% preamble ends
%% from now on, draft specific
%%%%%%%%%%%%%%%%%%%%%%%%%%%%%%%%%%%%
%% \RequirePackage[l2tabu, orthodox]{nag}
%%
\title{PatchSim Software Guide}
\author{Srini Venkatramanan\footnote{Email:srini@virginia.edu}}
\begin{document}
\maketitle

\section{Introduction}
Two common modeling frameworks for simulating epidemic dynamics are: ordinary 
differential equations (ODEs) and agent-based models (ABMs). While the former 
assumes a homogeneous mixing among the entire population, the latter uses a 
network to model the heterogeneous mixing among individuals. Though both 
methods have their own advantages, there is often a need for an approach with 
an intermediate level of complexity. Metapopulation models bridge the gap 
between ODEs and ABMs, since they model 
homogeneously mixing subpopulations connected by a network to model force of 
infection between subpopulations. See accompanying Model Description document 
for the underlying mathematical model.

PatchSim is a software code that allows for 
modeling SEIR dynamics across subpopulations (aka \emph{patches}), connected by 
a commuting-type flow network (aka \emph{travel matrix}). It has additional 
features such as vaccinations, stochastic vs deterministic mode, patch-specific 
parameterization, check-pointing. etc.

\section{User Guide}
\label{sec:user-guide}
The most recent version of PatchSim is maintained at 
\verb|https://github.com/srinivvenkat/PatchSim|. To use it, clone 
the directory, and run \verb|python test_det.py| inside the \verb|test/| 
folder. It should return no errors, and you should see two new files inside 
\verb|test/| namely, \verb|test_det.out| and \verb|test_det.log| \footnote{If 
you encounter any errors, contact srini@virginia.edu or file a Git issue}.

There are two basic test configurations (deterministic and stochastic) in the 
test folder. The deterministic test configuration (\verb|test/cfg_test_det|) 
looks as follows:
\begin{small}
\begin{center}
\begin{lstlisting}
PatchFile=test_pop.txt
NetworkFile=test_net.txt
NetworkType=Static

ExposureRate=0.65
InfectionRate=0.67
RecoveryRate=0.4
ScalingFactor=1

SeedFile=test_seed.txt
VaxFile=test_vax.txt
VaxDelay=4
VaxEfficacy=0.5

RandomSeed=12345
StartDate=1
Duration=100

OutputFile=test_det.out
OutputFormat=Whole
LogFile=test_det.log
\end{lstlisting}
\end{center}
\end{small}

There are five distinct sections in the configuration file (separated by empty 
lines for ease of comprehension). We will now take a closer look at each of the 
keywords and associated input files. 

\textit{Note:} All file paths are relative to the main code (in this case 
\verb|test/test.py|). All input files below are \emph{space 
separated} unless otherwise specified. No spaces allowed on either side of the 
`=' sign in the configuration file.

\subsection{Patch \& Network Info}
\begin{itemize}
	\item \verb|PatchFile| Contains patch population sizes in the following 
	format for each line:\\
	\verb|<patch_id> <population_size>|
	
	\item \verb|NetworkFile| Contains the travel matrix in the following 
	format for each line:\\
	\verb|<patch_id_i> <patch_id_j> <time_index> <Theta_i,j>|
	
	where \verb|<patch_id_i>| and \verb|<patch_id_j>| must be present in the 
	 \verb|PatchFile|. \verb|<time_index>| will take different values 
	depending on the \verb|NetworkType| described below. Note that 	
	PatchSim expects that $\Theta$ as described in the \verb|NetworkFile|
	obeys row stochasticity, failing which it may lead to errors or incorrect 
	outputs.
	
	\item \verb|NetworkType|: PatchSim currently accepts three different 
	NetworkTypes: \verb|Static|, \verb|Monthly|, \verb|Weekly|. Depending on 
	the NetworkType, the \verb|<time_index>| in NetworkFile takes the following 
	values.
	\begin{itemize}
		\item \verb|Static|: \verb|<time_index>| is always $0$.
		\item \verb|Monthly|: \verb|<time_index>| takes all values from 
		$[0,11]$ corresponding to month index (0 = Jan). 
		\item \verb|Weekly|: \verb|<time_index>| takes all values from 
		$[0,52]$ corresponding to weeks (0 = Jan 1-7).
	\end{itemize}  
\end{itemize}

All three keywords in this section are mandatory.


\subsection{Disease Parameters}
\begin{itemize}
	\item Disease parameters \verb|ExposureRate|, \verb|InfectionRate| and 
	\verb|RecoveryRate| respectively feed into $\beta$, $\alpha$ and $\gamma$ 
	respectively of the dynamics. All the three keywords are mandatory.
	
	\item \verb|ScalingFactor| is used to scale the daily infections before 
	printing to the output file. This can be used to represent any of the 
	following: (a) Reported infections, (b) Hospitalizations, (c) Emergency 
	department visits, (d) Deaths. This is an optional parameter, whose default 
	value is set as $1$.
\end{itemize}

\subsection{Seeding \& Vaccination}
\begin{itemize}
	\item \verb|SeedFile| Contains the (mandatory) seeding schedule in the 
	following format for each line:\\
	\verb|<day> <patch_id> <seed_cases>|
	
	\item \verb|VaxFile| Contains the (optional) vaccination schedule in the 
	following format for each line:\\
	\verb|<day> <patch_id> <vaccinations>|
	
	\item \verb|VaxDelay| and \verb|VaxEfficacy| represent the vaccine delay 
	(in days) and vaccine efficacy (as a probability). Both of these are 
	optional (default vaccine delay is $0$, default vaccine efficacy is $1$). 
	
\end{itemize}

\subsection{Simulation Time/Style}

\begin{itemize}
	\item \verb|RandomSeed| is an optional argument, which 
	when set automatically triggers stochastic mode of operation. It takes an 
	integer value. (Run \verb|python test/test_stoc.py| to test this feature.)
	
	\item \verb|StartDate| takes value from $[0,365]$ (0 = Jan 1) and denotes 
	the starting date (for the travel matrix)
	
	\item \verb|Duration| is the simulation duration in days (positive integer)
\end{itemize}

Both keywords are mandatory (although \verb|StartDate| is not used if 
\verb|NetworkType| is \verb|Static|). 

\subsection{Outputs}
\begin{itemize}
	\item \verb|OutputFile| is produced at the end of simulation. It contains 
	the epicurves of each patch in the following format for each line:\\
	\verb|<patch_id> <cases_0> <cases_1> ... <cases_T> |
	where $T$ is the duration of the epidemic.
	\item \verb|OutputFormat| (optional) specifies if PatchSim must produce 
	integer (\verb|Whole|) or floating point (\verb|Fractional|) values in 
	output time series. Default value is \verb|Whole|, which will produce 
	integer outputs (floor).
	\item \verb|LogFile| (optional) file for basic logging messages with 
	timestamp. 
\end{itemize}

\section{Advanced Features}
In addition the above listed attributes, PatchSim supports two additional 
advanced features described below:

\begin{itemize}
	\item \textbf{Checkpoints}: PatchSim allows you to load and save the state 
	array (values of S,E,I,R and V states for all patches). User can run a 
	simulation till day $T$, save state, and restart new simulation by loading 
	the saved state. This is accomplished by the following entries in the 
	config file:
	
	\begin{center}
		\begin{lstlisting}
		LoadState=True
		LoadFile=checkpoint1.npy
		SaveState=True
		SaveFile=checkpoint2.npy
		\end{lstlisting}
	\end{center}
	
	The above code, loads state from \verb|checkpoint1.npy|, runs the 
	simulation, and then saves final state to \verb|checkpoint2.npy|. When not 
	being used, \verb|LoadState| and \verb|SaveState| must be set to 
	\verb|False|.
	(Run \verb|python test/test_stopstart.py| to test this feature.)
	
	\item \textbf{Spatio-temporal parameters}: The disease parameters 
	\verb|ExposureRate|, \verb|InfectionRate| and \verb|RecoveryRate| are 
	usually listed in the Config file. If the users wishes to use heterogeneous values of \verb|ExposureRate|
	for specific patches, or have it vary with time,  then one can includes
	a \verb|ParamFile| attribute, pointing to a file.
	
	The file is space separated without a header line, with each line containing a patch id followed by either a single value (to be used for the entire simulation) or a list of $T$ values (where $T$ is the Duration of the simulation as specified in the Config file). Note that all patch ids present in the patch populations file must 
	be listed. (Run \verb|python test/test_paramfile.py| to test this feature and \verb|test/paramfile.txt| provides an example paramfile.)
	
\end{itemize}

\section{Limitations}
PatchSim is under constant development. The following are some of its current 
limitations, which we are hoping to address in subsequent releases.

\begin{itemize}
	\item \textbf{Flow matrix}: The current flow matrix interpretation is best 
	suited for commuting type datasets, and assumes the individual spends 
	\textbf{entire day} at the work patch. If you want to restrict to working 
	hours, the travel matrix needs to be appropriately scaled.
	
	For other types of flows (air travel, non-work activity trips), one needs 
	to explicitly process the travel matrix accounting for stay duration. 
	Further, population migration cannot be handled by this model. 

	\item \textbf{Disease Model}: Currently PatchSim supports only SEIR disease 
	model with vaccination. We hope to incorporate other disease models, 
	including vector-borne models soon.
	
\end{itemize}

Feedbacks are welcome. Any feature requests or bug reports can be filed filed as Git issues at: 
\verb|https://github.com/srinivvenkat/PatchSim|.

\section{Acknowledgments}
PatchSim development has been guided by inputs from the following folks at NSSAC/NDSSL (in alphabetical order of last name): 
Parantapa Bhattacharya, Jiangzhuo Chen, Arindam Fadikar, Sandeep Gupta, Bryan Lewis, Madhav Marathe, Anil Vullikanti.

\section*{Version History}
\begin{itemize}
	\item \textbf{23 Nov, 2017}: First version released.
	\item \textbf{27 Nov, 2017}: Ensured python3 compatibility.
	\item \textbf{21 Feb, 2018}: Checkpoints feature added.
	\item \textbf{23 Feb, 2018}: Patch-specific parameters allowed. 
	\item \textbf{08 Aug, 2018}: Stochastic mode (through RandomSeed) added.
	\item \textbf{06 Aug, 2019}: ParamFile now permits time-varying \verb|ExposureRate|. \verb|InfectionRate| and \verb|RecoveryRate| 
	cannot vary across patches/time. 
\end{itemize}

\end{document}


